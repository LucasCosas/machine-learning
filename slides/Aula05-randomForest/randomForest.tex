%
% Problemas para identifica��o de t�cnicas
% de representa��o de conhecimento
%
% by Fabr�cio Jailson Barth 2006
%

\documentclass[landscape,pdftex]{jomislides}

\slidesmag{5} % escala, qto maior maiores ser�o as letras/figras/etc.

%\centerslidesfalse

\usepackage{algorithmic}
\usepackage{alltt}
\usepackage{booktabs}
\usepackage{algorithm}

%
% Slides
% ======
%


\begin{document}

%\input{autorHeaders}

\title{Random Forest} 
\author{Fabr�cio Barth}
\institution{}
\date{Setembro de 2018}

\SlideHeader{}
            {}
\SlideFooter{\theslidepartheading $\;$ --- $\;$ \theslideheading}
            {\theslide}

\vpagecolor[white]{white}


\subtitle{}

\maketitle

\begin{Slide}{Ensemble Learning}
\begin{itemize}
	\item M�todos que geram diversos modelos e agregam o seu resultado.
	\item No caso do Random Forest, s�o geradas diversas �rvores e cada �rvore � gerada considerando apenas um sub-conjunto do conjunto de treinamento.
\end{itemize}
\end{Slide}

\begin{Slide}{Random Forest}
\begin{itemize}
	\item O algoritmo possui apenas dois par�metros configur�veis: 
	\begin{itemize}
		\item quantidade de atributos considerados em cada �rvore ($m_{try}$), e;
		\item quantidade de �rvores ($n_{tree}$).
	\end{itemize}
\end{itemize}
\end{Slide}

\begin{Slide}{Random Forest}
Para problemas de classifica��o e regress�o o algoritmo funciona da seguinte forma:
\begin{itemize}
	\item Cria $n_{tree}$ sub-conjuntos de exemplos a partir do dataset original.
	\item Para cada sub-conjunto de exemplos cria-se uma �rvore de classifica��o ou regress�o sem poda. 
	A cria��o de cada �rvore considera apenas um sub-conjunto de exemplos: $m_{try}$ atributos selecionados aleatoriamente e $2/3$ dos exemplos tamb�m selecionados aleatoriamente. 
	
	\newpage
	
	\item A predi��o para novos dados acontece pela agrega��o das predi��es das $n_{tree}$ �rvores.
	\item Para problemas de \textbf{classifica��o} � considerado a maioria dos votos.
	\item Para problemas de \textbf{regress�o} � considerado a m�dia dos votos.  
	
\end{itemize}	
\end{Slide}

\begin{Slide}{Estimativa de erro}
	\begin{itemize}
		\item Uma estimativa de erro, usando apenas o conjunto de treinamento, pode ser obtida atrav�s do conjunto de treinamento. Ao inv�s de ser utilizado algum outro m�todo, como \textit{cross-validation}.
		\item Para cada �rvore constru�da � usado um sub-conjunto de exemplos. $1/3$ dos exemplos s�o mantidos fora do conjunto de treinamento. Estes exemplos mantidos fora do conjunto de treinamento s�o utilizados como teste.
	\end{itemize}
\end{Slide}

\begin{Slide}{Exemplo}
	http://rpubs.com/fbarth/exemploRandomForest
\end{Slide}


\begin{Slide}{Material de \textbf{consulta}}
  \begin{itemize}
  \item Liaw and Wiener. Classification and Regression by randomForest. R News 2 (3): 18--22 (2002)
  \item Breiman and Cutler. Random Forests. Acessado em https://www.stat.berkeley.edu/~breiman/RandomForests/
  \end{itemize}
\end{Slide}


\end{document}

